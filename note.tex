\documentclass[11pt,a4paper]{amsart}

%Fonts

\usepackage{libertine}
\usepackage{euler}
\usepackage[T1]{fontenc}

%Packages.

\usepackage{verbatim}
\usepackage{amsmath}
\usepackage{amssymb}
%\usepackage{amsfonts}
\usepackage{amsthm}
\usepackage{mathrsfs}
\usepackage{tikz}
\usetikzlibrary{matrix,arrows}
\usepackage[all]{xy}
\usepackage{mathrsfs}
\usepackage{a4wide}
\usepackage{enumerate}
\usepackage{hyperref}
%\usepackage{showkeys}
\usepackage{setspace}

%Theorem styles.

\theoremstyle{plain}
\newtheorem{conj}{Conjecture}
\newtheorem{theorem}{Theorem}%[section]
\newtheorem{lemma}[theorem]{Lemma}%[section]
\newtheorem{proposition}[theorem]{Proposition}%[section]
\newtheorem{corollary}[theorem]{Corollary}%[section]
\newtheorem{claim}[theorem]{Claim}%[section]
\newtheorem{question}[theorem]{Question}
\newtheorem{conjecture}[theorem]{Conjecture}
\newtheorem*{thm}{Theorem}

\theoremstyle{definition}%[section]
\newtheorem{definition}[theorem]{Definition}%[section]
\newtheorem{example}[theorem]{Example}%[section]
\newtheorem{project}{Project}

\theoremstyle{remark}%[section]
\newtheorem{remark}[theorem]{Remark}%[section]

%Title etc.

\title{Coarse embedding certain spaces of large girth into groups.} 
%\date{}
\author{Martin Finn-Sell}

\begin{document}

\maketitle

\section{Introduction.}

\section{Partial actions and coarse embeddability.}

A partial bijection is...
A map of a group into the set of partial bijections is called a dual prehomomorphism if...

\begin{definition}
Let $\Gamma$ be a finitely generated discrete group and let $X$ be a (locally compact Hausdorff) topological space. A \textit{partial action} of $\Gamma$ on $X$ is a dual prehomomorphism that has the following properties:
\begin{enumerate}
\item The domain $D_{\theta_{g}^{*}\theta_{g}}$ is an open set for every $g$.
\item $\theta_{g}$ is a continuous map.
\item The union: $\bigcup_{g \in G}D_{\theta_{g}^{*}\theta_{g}}$ is $X$.
\end{enumerate}
We say that the action is \textit{transitive} if for every pair of points $(x,y) \in X^{2}$ there is a $g\in \Gamma$ such that $\theta_{g}(x)=y$ and \textit{free} if for every point, the set $\lbrace g \in \Gamma | \theta_{g}(x)=x \rbrace=\lbrace 1_{\Gamma} \rbrace$.
\end{definition}

The advantage of studying partial actions is the additional flexibility of working with partial bijections. However it is often useful to convert this data back into something tangible with regards to the group. In this manner, we introduce the globalisation; that is a space on which the group acts that extends the original partial action.

\begin{definition}
A \textit{globalisation} of a partial action $\theta: G \rightarrow I(X)$ is a space $Y$ with an injection $X \hookrightarrow Y$ and action $\tilde{\theta}$ of $G$ such that the partial action obtained from restricting the action $\tilde{\theta}$ to $X$ is equal to $\theta$. 
\end{definition}

A globalisation is minimal if it injects into any other globalisation. In \cite{MR2041539} the authors proved that for any partial action of a group $G$ there is a unique globalisation (up to equivalence of partial actions). This is defined as follows:

\begin{definition}
Let $X$ be a topological space and let $G$ be a group acting partially on $X$. Then we denote by $\Omega$ the \textit{Morita envelope} of the action of $G$ on $X$, which is constructed as follows:

Consider the space $X\times G$, equipped with the product topology. Then define $\sim$ on $X\times G$ by $(x,g)\sim (y,h)$ if and only if $x(h^{-1}g)=y$. We define $\Omega$ as the quotient of $X\times G$ by $\sim$ with the quotient topology. 

$G$ acts on $\Omega$ using right multiplication by inverses on the group factor of the equivalence classes. Clearly the map that sends $x \in X$ to $[1,x] \in \Omega$ is a topological injection. The main result of \cite{MR2041539} is that this new topological space is minimal amongst globalisations of $X$.
\end{definition}

One useful outcome of globalization of the partial action is to get an idea of the structure of the original space.

\begin{theorem}
Transitive globalisations of discrete group actions on discrete spaces are quotients of the acting group by stabiliser subgroups.
\end{theorem}

\begin{corollary}
If the action is free then the stabiliser subgroups are trivial; so the globalisation is the group itself.
\end{corollary}

Remark that given a space with a partial action that is free and transitive, we certainly get an embedding of the space into the group in question. The idea is to show that this map must be coarse. We give related ideas from coarse geometry.
\begin{definition}
Let $X$ be a set and let $\mathcal{E}$ be a collection of subsets of $X \times X$. If $\mathcal{E}$ has the following properties:
\begin{enumerate}
\item $\mathcal{E}$ is closed under finite unions;
\item $\mathcal{E}$ is closed under taking subsets;
\item $\mathcal{E}$ is closed under the induced product and inverse that comes from the groupoid product on $X \times X$.
\item $\mathcal{E}$ contains the diagonal
\end{enumerate}
Then we say $\mathcal{E}$ is a \textit{coarse structure} on $X$ and we call the elements of $\mathcal{E}$ \textit{entourages}. If in addition $\mathcal{E}$ contains all finite subsets then we say that $\mathcal{E}$ is \textit{weakly connected}.
\end{definition}

For a given family of subsets $\mathcal{S}$ of $X \times X$ we can consider the smallest coarse structure that contains $\mathcal{S}$. This is the coarse structure generated by $\mathcal{S}$. We can use this to give some examples of coarse structures.

\begin{definition}
Let $X$ be a coarse space with a coarse structure $\mathcal{E}$ and consider $\mathcal{S}$ a family of subsets of $\mathcal{E}$. We say that $\mathcal{E}$ is generated by $\mathcal{S}$ if every entourage $E \in \mathcal{E}$ is contained in a finite union of subsets of $\mathcal{S}$.
\end{definition}

\begin{example}\label{ex:MCS}
Let $X$ be a metric space. Then consider the collection $\mathcal{S}$ given by the $R$-neighbourhoods of the diagonal in $X\times X$; that is, for every $R>0$ the set:
\begin{equation*}
\Delta_{R}=\lbrace (x,y) \in X \times X | d(x,y)\leq R \rbrace
\end{equation*}
Then let $\mathcal{E}$ be the coarse structure generated by $\mathcal{S}$. This is called the \textit{metric coarse structure} on $X$. It is a uniformly locally finite proper coarse structure that is weakly connected when $X$ is a uniformly discrete bounded geometry (proper) metric space.
\end{example}

\begin{example}\label{ex:GACS}
Let $G$ be a group and let $X$ be a right $G$-set. Define:
\begin{equation*}
\Delta_{g}=\lbrace (x,x.g) | x \in X \rbrace  
\end{equation*}
We call the coarse structure generated by the family $\mathcal{S}:=\lbrace \Delta_{g} | g\in G\rbrace$ the \textit{group action coarse structure} on $X$. If $X$ is not a transitive $\Gamma$-space the group action coarse structure will not be weakly connected.
\end{example}

In the situation that $X$ admits a transitive $G$-action by translations, the group action coarse structure generates a substructure of the metric coarse structure. If additionally, each $\Delta_{R}$ is contained in finitely many $\Delta_{g}$, then the group action coarse structure will be the same as the metric coarse structure.

\begin{theorem}
If a metric space $X$ admits a free, transitive partial action of a discrete group $\Gamma$ such that for each $R>0$ there are $g_{1},...,g_{n_{R}} \in \Gamma$ such that $\Delta_{R}(X) \subseteq \bigcup_{i=1}^{n_{R}}\theta_{g_{i}}$ then the map $i$ from globalisation theorem is a coarse embedding.
\end{theorem}

\section{A strengthening of this result.}

Let $S$ be a semigroup and let $s \in S$. Then $u$ is said to be an \textit{inverse} for $s$ if $sus=s$ and $usu=u$. A semigroup $S$ is \textit{regular} if every element $s$ has some inverse element $u$, and \textit{inverse} if that inverse element is unique, in which case we denote it by $s^{*}$. Groups are examples of semigroups that are inverse with only a single idempotent. It is clear that in an inverse semigroup $S$ every element $ss^{*}$ is idempotent, and this classifies the structure of idempotents \cite{}. We remark also that the idempotents of $S$ form a commutative inverse subsemigroup $E(S)$ that can be partially ordered using this multiplication: Let $e,f \in E(S)$ then:
\begin{equation*}
e \leq f \Leftrightarrow ef=e
\end{equation*}
is a partial order on $E(S)$ that extends to $S$ using:
\begin{equation*}
s \leq t \Leftrightarrow (\exists e \in E) et=s
\end{equation*}
If $s$ and $t$ are partial bijections on some set $X$ this describes precisely when $s$ is a restriction of $t$ to some subset of $dom(t)$.

This order allows us to define a more rigid and accessible class of inverse semigroup:


\begin{definition}
Let $z \in S$. We say $z$ is a zero element if $z \in E(S)$ and $zs=sz=z$. The empty partial bijection is an example of such a zero element. With this in mind we say $S$ is 0-E-unitary if $\forall e \in E\setminus 0, s \in S$ $e \leq s$ implies $s \in E$.
\end{definition}

For concrete semigroups of partial bijections this condition is equivalent to the restriction of any element never being an idempotent.

Let $S$ be a inverse semigroup. Then it is possible to define a \text{universal group} of $S$ \cite{}, denoted by $U(S)$, to be the group generated by the elements of $S$ with relations $s\cdot t = st$ if $st \not = 0$. Clearly, there is a map $\Phi$ from $S$ to the universal group $U(S)$, given by mapping $s$ to its corresponding symbol in $U(S)$. This map is \textit{not} a homomorphism, but after adjoining a zero element to $U(S)$ does satisfy the inequality $\Phi(st) \leq \Phi(s)\Phi(t)$. Such a map is called a \textit{prehomomorphism}. 

\begin{definition}
We say $S$ is 0-F-inverse if the preimage of each group element in $U(S)$ has a maximal element within the partial order of $S$. In this case, we denote these elements by $Max(S)$, and the universal group is generated by the set $Max(S)$ with the product: $s\ast t = u$, where $u$ is the unique maximal element above $st \in S$. 

Finally, $S$ is said to be strongly $0$-F-inverse if it is $0$-F-inverse and there is some group $\Gamma$ and a $0$-restricted idempotent pure homomorphism onto $G^{0}$. In particular, this must factor though the universal group, so this amounts to asking if the map $\Phi$ is idempotent pure.
\end{definition}

In general, universal groups are hard to calculate. They play a role in determining the structure of this class of inverse monoid however \cite{}. This class of inverse monoid is also intimately connected to partial actions of discrete groups. We need an intermediate object called the Birget-Rhodes expansion of a group $\Gamma$.

\begin{example}
In \cite{MR745358,MR2221438} a inverse monoid was introduced that is universal for dual prehomomorphisms from a general inverse semigroup. In the context of a group $\Gamma$ this is called the \textit{prefix expansion}; its elements are given by pairs: $(X,g)$ for $\lbrace 1,g\rbrace \subset X$, where $X$ is a finite subset of $\Gamma$. The set of such $(X,g)$ is then equipped with a product and inverse:
\begin{equation*}
(X,g)(Y,h) = (X\cup gY,gh)\mbox{ , } (X,g)^{-1}=(g^{-1}X,g^{-1})
\end{equation*}
This has maximal group homomorphic image $\Gamma$, and it has the universal property that it is the largest such inverse monoid. We denote this by $\Gamma^{Pr}$. The partial order on $\Gamma^{Pr}$ can be described by reverse inclusion, induced from reverse inclusion on finite subsets of $\Gamma$. It is F-inverse, with maximal elements: $\lbrace(\lbrace 1,g \rbrace, g):g \in \Gamma \rbrace$.
\end{example}

\begin{proposition}\label{Prop:Strongly}
Let $S = \langle \theta_{g} | g \in \Gamma \rangle$, where $\theta: \Gamma \rightarrow S$ is a partial action. If $S$ is 0-F-inverse with $Max(S) = \lbrace \theta_{g} | g \in \Gamma \rbrace$ where each nonzero $\theta_{g}$ is not idempotent when $g \not = e$ then $S$ is strongly 0-F-inverse; it has a $0$-restricted idempotent pure homomorphism onto $\Gamma$.
\end{proposition}
\begin{proof}
We build a map $\Phi$ back onto $G^{0}$. Let $m: S\setminus \lbrace 0 \rbrace \rightarrow Max(S)$ be the map that sends each non-zero $s$ to the maximal element $m(s)$ above $s$ and consider the following diagram:
\begin{equation*}
\xymatrix{
G\ar@{->}[r]^{\theta}\ar@{->}[dr]^{}  & S\ar@{->}[dr]^{\Phi}  & \\
  & G^{pr} \ar@{->}[r]^{\sigma}\ar@{->}[u]^{\overline{\theta}}  & G^{0}
}
\end{equation*}
where $G^{pr}$ is the prefix expansion of $G$. Define the map $\Phi:S \rightarrow G^{0}$ by:
\begin{equation*}
\Phi(s)=\sigma ( m ( \overline{\theta}^{-1} (m(s)))), \Phi(0)=0
\end{equation*}
For each maximal element the preimage under $\overline{\theta}$ is well defined as the map $\theta_{g}$ has the property that $\theta_{g}=\theta_{h} \Rightarrow g=h$ precisely when $\theta_{g} \not = 0 \in S$. Given the preimage is a subset of the F-inverse monoid $G^{pr}$ we know that the maximal element in the preimage is the element $(\lbrace 1,g \rbrace,g)$ for each $g \in G$, from where we can conclude that the map $\sigma$ takes this onto $g \in G$.

We now prove it is a prehomomorphism. Let $\theta_{g},\theta_{h} \in S$, then:
\begin{eqnarray*}
\Phi(\theta_{g})=\sigma ( m(\overline{\theta}^{-1}(\theta_{g}))) = \sigma ( \lbrace 1,g \rbrace, g)= g\\
\Phi(\theta_{h})=\sigma ( m(\overline{\theta}^{-1}(\theta_{h}))) = \sigma ( \lbrace 1,h \rbrace, h)= h\\
\Phi(\theta_{gh})=\sigma ( m(\overline{\theta}^{-1}(\theta_{gh}))) = \sigma ( \lbrace 1,gh \rbrace, gh)= gh
\end{eqnarray*}
Hence whenever $\theta_{g},\theta_{h}$ and $\theta_{gh}$ are defined we know that $\Phi(\theta_{g}\theta_{h})=\Phi(\theta_{g})\Phi(\theta_{h})$. They fail to be defined if:
\begin{enumerate}
\item If $\theta_{gh} = 0$ in $S$ but $\theta_{g}$ and $\theta_{h} \not = 0$ in $S$, then $0=\Phi(\theta_{g}\theta_{h})\leq \Phi(\theta_{g})\Phi(\theta_{h})$

\item If (without loss of generality) $\theta_{g}=0$ then $0=\Phi(0.\theta_{h})= 0.\Phi(\theta_{h})=0$
\end{enumerate}
So prove that the inverse monoid $S$ is strongly 0-F-inverse it is enough to prove then that the map $\Phi$ is idempotent pure, and without loss of generality it is enough to consider maps of only the maximal elements - as the dual prehomomorphism property implies that in studying any word that is non-zero we will be less than some $\theta_{g}$ for some $g \in G$.

So consider the map $\Phi$ applied to a $\theta_{g}$:
\begin{equation*}
\Phi(\theta_{g})=\sigma ( m(\overline{\theta}^{-1}(\theta_{g}))) = \sigma ( \lbrace 1,g \rbrace, g)= g
\end{equation*}

Now assume that $\Phi(\theta_{g}) = e_{G}$. Then it follows that $\sigma (m (\overline{\theta}^{-1}(\theta_{g})))=e_{G}$. As $\sigma$ is idempotent pure, it follows then that $m(\overline{\theta}^{-1}(\theta_{g}))=1$, hence for any preimage $t\in \theta}^{-1}(\theta_{g})$ we know that $t \leq 1$, and by the property of being 0-E-unitary it then follows that $t \in E(G^{pr})$. Mapping this back onto $\theta_{g}$ we can conclude that $\theta_{g}$ is idempotent, but by assumption this only occurs if $g = e$.\end{proof}

\begin{corollary}
Let $S = \langle \theta_{g} | g \in \Gamma \rangle$, where $\theta: \Gamma \rightarrow S$ is a free partial action. If $S$ is 0-F-inverse with maximal elements $Max(S) = \lbrace \theta_{g} | g \in \Gamma \rbrace$ then $S$ is strongly $0$-F-inverse.\qed
\end{corollary}

\begin{definition}
A abstract partial translation structure on a set $X$ is...

A translation structure is a ATS on a metric space that generates the metric coarse structure.

A partial translation structure is called strong if...
\end{definition}

\begin{theorem}\label{Thm:SPTSCE}
If $X$ is metric space with a strong partial translation structure then it coarsely embeds into a group (the universal group).
\end{theorem}

\begin{remark}
This actually characterises coarse embeddability into groups; see Theorem 29 \cite{}.
\end{remark}

\section{Embedding spaces of graphs into groups.}

\begin{definition}

\end{definition}

\subsection{Partial actions from labellings.}

\begin{lemma}
Pedersens lemma.
\end{lemma}

Quotients of a free group always have a Pedersen labelling. 

\begin{definition}
Coherent and Pedersen labelling.
\end{definition}

\subsection{A construction of a group from a labelling.}

Give the construction of the group.

\begin{theorem}
Given a sequence of finite graphs $\lbrace X_{i} \rbrace$ such that the space of graphs $X$ has a coherent Pedersen labelling. Then the group $\Gamma$ constructed above acts freely partially on $X$.
\end{theorem}

We need a trick to turn a free partial action into strong partial translation structure.

\begin{proposition}\label{Prop:CheapTrick}
Given a free partial action of a discrete group $\Gamma$ on $X$ there is a strong partial translation structure $\mathcal{T}$ on $X$ built from the action that has universal group $\Gamma\ast F_{\infty}$.
\end{proposition}

This allows us to complete a labelling into a constructible coarse embedding by adding transitivity and applying Theorem \ref{Thm:SPTSCE}

\begin{theorem}
Let $X$ be a space of graphs with a coherent Pedersen Labelling. Then $X$ coarsely embeds into $\Gamma \ast F_{\infty}$.
\end{theorem}

Over the next section we outline the tools to apply this general result to a particular space with large girth.

\section{The construction.}

\subsection{The space of Arzantseva, Guentner and Spakula.}

Some words on their construction, explain why the structure is nice, how the girth grows and give our seed graph.

\subsection{A set of small cancellation words of Erchler-Osin.}

Explain their construction, state their theorem and the forms of words we'll be using.

\subsection{All together now...}

Outline how we'll be labelling by first breaking into disjoint short cycles. Prove this is compatible.

\begin{lemma}
The labelling of the short cycles above gives rise to a well defined labelling of the entire sequence.
\end{lemma}

Now we adjust the labelling such that it is both coherent and Pedersen.

\begin{lemma}
By swapping certain letters where necessary, the labelling becomes Pedersen.
\end{lemma}

\begin{lemma}
With the modifications above the labelling is coherent.
\end{lemma}

\begin{theorem}
The group $\Gamma$ is not exact.
\end{theorem}


\bibliographystyle{plain}
\bibliography{ref.bib}


\end{document}